\documentclass[10.5pt]{jsarticle}
\usepackage[hiresbb]{graphicx}
\usepackage{color}
\usepackage{amsmath,amssymb}
\usepackage{here}

\begin{document}
\section{目的}
機械学習を実装することについて勉強しながら、私自身の新たな可能性を見つける。

\section{概説}
本レポートは人間のピアノ学習を模倣し、jazzセクションにおいて{\bf 教師が弾いたフレーズ}と{\bf コード進行}から
機械は音楽フレーズの作り方を学習できる可能性について研究するものとする。

将来の実用化と音楽のユーザーインターフェースを考慮し、プラットホームをiOSアプリ、プログラム言語をSwiftと選定した。
本プロジェクトのプログラムコードは全てGithub上で一般公開する:https://github.com/huyuu/MLMusicComposition

まず価値関数Qの環境部分については、複数の小節(Measure)で構成され、各小節の中では一つの和音(Cord)と複数の音符(Note)が存在し、
それぞれの音符は音価(NoteValue: 単純なString型)と長さ(Duration: Int型)で表現されている。


\section{}


\end{document}
